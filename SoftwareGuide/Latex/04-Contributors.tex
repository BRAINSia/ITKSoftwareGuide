\chapter*{Contributors}
\noindent

The Insight Toolkit \href{http://www.itk.org}{(ITK)} has been created by the
efforts of many talented individuals and prestigious organizations. It is also
due in great part to the vision of the program established by Dr. Terry Yoo
and Dr. Michael Ackerman at the National Library of Medicine.

This book lists a few of these contributors in the following paragraphs. Not
all developers of ITK are credited here, so please visit the Web pages at
\href{http://www.itk.org/HTML/About.htm}{http://www.itk.org/HTML/About.htm} 
for the names of additional contributors, as well as checking the GIT source
logs for code contributions.

The following is a brief description of the contributors to this software
guide and their contributions.


{\bf Luis Ib\'{a}\~{n}ez} is principal author of this text.
He assisted in the design and layout of the text, implemented the bulk of
the \LaTeX{} and CMake build process, and was responsible for the bulk of 
the content. He also developed most of the example code found in the
\code{Insight/Examples} directory.

{\bf Will Schroeder} helped design and establish the organization 
of this text and the \code{Insight/Examples} directory. He is principal 
content editor, and has authored several chapters.

{\bf Lydia Ng} authored the description for the registration framework
and its components, the section on the multiresolution framework, and
the section on deformable registration methods. She also edited the
section on the resampling image filter and the sections on various
level set segmentation algorithms.

{\bf Joshua Cates} authored the iterators chapter and the text and examples
describing watershed segmentation. He also co-authored the level-set
segmentation material.

{\bf Jisung Kim} authored the chapter on the statistics framework.

{\bf Julien Jomier} contributed the chapter on spatial objects and examples on
model-based registration using spatial objects.

{\bf Karthik Krishnan} reconfigured the process for automatically generating
images from all the examples. Added a large number of new examples and updated
the Filtering and Segmentation chapters for the second edition.

{\bf Stephen Aylward} contributed material describing spatial objects and
their application.

{\bf Tessa Sundaram} contributed the section on deformable registration using
the finite element method.

{\bf YinPeng Jin} contributed the examples on  hybrid segmentation methods. 

{\bf Celina Imielinska} authored the section describing the principles of
hybrid segmentation methods.

{\bf Mark Foskey} contributed the examples on the
AutomaticTopologyMeshSource class.

{\bf Mathieu Malaterre} contributed the entire section on the description and
use of DICOM readers and writers based on the GDCM library. He also contributed
an example on the use of the VTKImageIO class.

{\bf Gavin Baker} contributed the section on how to write composite filters.
Also known as minipipeline filters.

Since the software guide is generated in part from the ITK source code
itself, many ITK developers have been involved in updating and
extending the ITK documentation.  These include {\bf David Doria},
{\bf Bradley Lowecamp}, {\bf Mark Foskey}, {\bf Ga\"{e}tan Lehmann},
{\bf Andreas Schuh}, {\bf Tom Vercauteren}, {\bf Cory Quammen}, {\bf Daniel Blezek},
{\bf Paul Hughett}, {\bf Matt McCormick}, {\bf Josh Cates}, {\bf Arnaud Gelas},
{\bf Jim Miller}, {\bf Brad King}, {\bf Gabe Hart}, {\bf Hans Johnson}.

{\bf Hans Johnson}, {\bf Kent Williams}, {\bf Constantine Zakkaroff}, and {\bf
Matt McCormick} updated the documentation for ITK Version 4.



