\usepackage[dvips]{graphicx}
\usepackage{times}
\usepackage{color}
\usepackage{listings}
\usepackage{minted}
\usepackage{setspace}
\usepackage{hyphenat}
\usepackage[toc,page]{appendix}
\usepackage{verbatim}

\usepackage{draftwatermark}
%\usepackage[firstpage]{draftwatermark}
\SetWatermarkText{DRAFT-ITK}%default is DRAFT
\SetWatermarkLightness{0.90}%default is 0.8
\SetWatermarkScale{0.8}%default is 1.2

%\allowhyphens

\lstnewenvironment{itklisting}
{\singlespacing\lstset{language=C++}}
{}

\definecolor{ltgray}{rgb}{0.97,0.97,0.97}
\usemintedstyle{emacs}

\lstset{
         basicstyle=\footnotesize\ttfamily, % Standardschrift
         %numbers=left,               % Ort der Zeilennummern
         numberstyle=\tiny,          % Stil der Zeilennummern
         %stepnumber=2,               % Abstand zwischen den Zeilennummern
         numbersep=2pt,              % Abstand der Nummern zum Text
         tabsize=2,                  % Groesse von Tabs
         extendedchars=true,         %
         breaklines=true,            % Zeilen werden Umgebrochen
         keywordstyle=\color{red},
 %    frame=b,
 %        keywordstyle=[1]\textbf,    % Stil der Keywords
 %        keywordstyle=[2]\textbf,    %
 %        keywordstyle=[3]\textbf,    %
 %        keywordstyle=[4]\textbf,   \sqrt{\sqrt{}} %
         stringstyle=\color{black}\ttfamily, % Farbe der String
         showspaces=false,           % Leerzeichen anzeigen ?
         showtabs=false,             % Tabs anzeigen ?
         xleftmargin=17pt,
         framexleftmargin=17pt,
         framexrightmargin=5pt,
         framexbottommargin=4pt,
         backgroundcolor=\color{ltgray},
         showstringspaces=false      % Leerzeichen in Strings anzeigen ?
 }

 \lstloadlanguages{% Check Dokumentation for further languages ...
         %[Visual]Basic
         %Pascal
         %C
         C++
         %XML
         %HTML
         %Java
 }
%\DeclareCaptionFont{blue}{\color{blue}}

%\captionsetup[lstlisting]{singlelinecheck=false, labelfont={blue}, textfont={blue}}
\usepackage{caption}
\DeclareCaptionFont{white}{\color{white}}
\DeclareCaptionFormat{listing}{\colorbox[cmyk]{0.43, 0.35, 0.35,0.01}{\parbox{\textwidth}{\hspace{15pt}#1#2#3}}}
\captionsetup[lstlisting]{format=listing,labelfont=white,textfont=white, singlelinecheck=false, margin=0pt, font={bf,footnotesize}}

\newif\ifitkFullVersion
\itkFullVersiontrue
%\itkFullVersionfalse


%%%%%%%%%%%%%%%%%%%%%%%%%%%%%%%%%%%%%%%%%%%%%%%%%%%%%%%%%%%%%%%%%%%
%
%
%   Load configuration parameters prepared by CMake
%
%
%%%%%%%%%%%%%%%%%%%%%%%%%%%%%%%%%%%%%%%%%%%%%%%%%%%%%%%%%%%%%%%%%%%

\input{SoftwareGuideConfiguration.tex}

%%%%%%%%%%%%%%%%%%%%%%%%%%%%%%%%%%%%%%%%%%%%%%%%%%%%%%%%%%%%%%%%%%
%
%  hyperref should be the last package to be loaded.
%
%%%%%%%%%%%%%%%%%%%%%%%%%%%%%%%%%%%%%%%%%%%%%%%%%%%%%%%%%%%%%%%%%%
\ifitkPrintedVersion
\usepackage[dvips,
pdftitle={ITK Software Guide},
pdfauthor={Hans Johnson and Luis Ib'{a}~{n}ez and Matthew McCormick and the Insight Software Consortium},
pdfsubject={Medical Image Segmentation and Registration Toolkit}
pdfkeywords={Registration,Segmentation,Guide},
pdfpagemode={UseOutlines},
bookmarks,bookmarksopen,
pdfstartview={FitH},
backref,
colorlinks,linkcolor={black},citecolor={black},urlcolor={black},
]{hyperref}
\else
\usepackage[dvips,
pdftitle={ITK Software Guide},
pdfauthor={Hans Johnson and Luis Ib'{a}~{n}ez and Matthew McCormick and the Insight Software Consortium},
pdfsubject={Medical Image Segmentation and Registration Toolkit},
pdfkeywords={Registration,Segmentation,Guide},
pdfpagemode={UseOutlines},
bookmarks,bookmarksopen,
pdfstartview={FitH},
backref,
colorlinks,linkcolor={blue},citecolor={blue},urlcolor={blue},
]{hyperref}
\fi

%%%%%%%%%%%%%%%%%%%%%%%%%%%%%%%%%%%%%%%%%%%%%%%%%%%%%%%%%%%%%%%%%%%
%
%
%           The Insight Toolkit Software Guide
%
%
%%%%%%%%%%%%%%%%%%%%%%%%%%%%%%%%%%%%%%%%%%%%%%%%%%%%%%%%%%%%%%%%%%%

\author{Hans J. Johnson, Matthew M. McCormick, Luis Ib\'{a}\~{n}ez, and the \emph{Insight Software Consortium}}

\authoraddress{
  \url{http://itk.org}\\
  Email: \email{community@itk.org}
}

\date{\today}

% actually write the .idx file
\makeindex

\setcounter{tocdepth}{3}
