%%%%%%%%%%%%%%%%%%%%%%%%%%%%%%%%%%%%%%%%%%%%%%%%%%%%%%%%%%%%%%%
%
%
%   This file is included in Registration.tex
%
%   Lablels and section entries are defined in that file.
%
%
%
%%%%%%%%%%%%%%%%%%%%%%%%%%%%%%%%%%%%%%%%%%%%%%%%%%%%%%%%%%%%%%%

For the problem of intra-modality deformable registration, the Insight
Toolkit provides an implementation of Thirion's ``demons'' algorithm
\cite{Thirion1995b,Thirion1998}. 
In this implementation, each image is viewed as a set of iso-intensity
contours.  The main idea is that a regular grid of forces deform an image by
pushing the contours in the normal direction.  The orientation and magnitude
of the displacement is derived from the instantaneous optical flow equation:

\begin{equation}
\bf{D}(\bf{X}) \cdot \nabla f(\bf{X}) = - \left(m(\bf{X}) - f(\bf{X}) \right)
\label{eqn:OpticalFlow}
\end{equation}

In the above equation, $f(\bf{X})$ is the fixed image, $m(\bf{X})$
is the moving image to be registered, and $\bf{D}(\bf{X})$ is the displacement 
or optical flow between the images. It is well known in optical flow
literature that Equation \ref{eqn:OpticalFlow} is insufficient to specify 
$\bf{D}(\bf{X})$ locally and is usually determined using some form of
regularization. For registration, the projection of the vector on the
direction of the intensity gradient is used:

\begin{equation}
\bf{D}(\bf{X}) = - \frac
{\left(  m(\bf{X}) - f(\bf{X}) \right) \nabla f(\bf{X})}
{\left\|  \nabla f \right\|^2 } 
\end{equation}

However, this equation becomes unstable for small values of the image gradient,
resulting in large displacement values. To overcome this problem, Thirion
re-normalizes the equation such that:

\begin{equation}
\bf{D}(\bf{X}) = - \frac
{\left(  m(\bf{X}) - f(\bf{X}) \right) \nabla f(\bf{X})}
{\left\|  \nabla f \right\|^2 + \left(  m(\bf{X}) - f(\bf{X}) \right)^2 / K } 
\label{eqn:DemonsDisplacement}
\end{equation}

Where $K$ is a normalization factor that accounts for the units imbalance
between intensities and gradients. This factor is computed as the mean squared
value of the pixel spacings. The inclusion of $K$ make the force computation to
be invariant to the pixel scaling of the images.

Starting with an initial deformatin field $\bf{D}^{0}(\bf{X})$, the demons
algorithm iteratively updates the field using Equation
\ref{eqn:DemonsDisplacement} such that the field at the $N$-th iteration is
given by:

\begin{equation}
\bf{D}^{N}(\bf{X}) = \bf{D}^{N-1}(\bf{X}) - \frac
{\left(  m(\bf{X}+ \bf{D}^{N-1}(\bf{X})) 
- f(\bf{X}) \right) \nabla f(\bf{X})}
{\left\|  \nabla f \right\|^2 + \left(  
m(\bf{X}+ \bf{D}^{N-1}(\bf{X}) )
 - f(\bf{X}) \right)^2 } 
\label{eqn:DemonsUpdateEquation}
\end{equation}

Reconstruction of the deformation field is an ill-posed problem where
matching the fixed and moving images has many solutions. For example, since
each image pixel is free to move independently, it is possible that all
pixels of one particular value in $m(\bf{X})$ could map to a single image
pixel in $f(\bf{X})$ of the same value. The resulting deformation field may
be unrealistic for real-world applications. An option to solve for the field
uniquely is to enforce an elastic-like behavior, smoothing the deformation
field with a Gaussian filter between iterations.

In ITK, the demons algorithm is implemented as part of the finite difference
solver (FDS) framework and its use is demonstrated in the following example.

\input{DeformableRegistration2.tex} 

A variant of the force computation is also implemented in which the gradient of
the deformed moving image is also involved. This provides a level of symmetry
in the force calculation during one iteration of the PDE update. The equation
used in this case is

\begin{equation}
\bf{D}(\bf{X}) = - \frac
{2 \left(  m(\bf{X}) - f(\bf{X}) \right) \left(  \nabla f(\bf{X}) +  \nabla g(\bf{X}) \right) }
{\left\|  \nabla f + \nabla g \right\|^2 + \left(  m(\bf{X}) - f(\bf{X}) \right)^2 / K } 
\label{eqn:DemonsDisplacement2}
\end{equation}

The following example illustrates the use of this defomable registration
method.

\input{DeformableRegistration3.tex} 



