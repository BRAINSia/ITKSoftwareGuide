%%%%%%%%%%%%%%%%%%%%%%%%%%%%%%%%%%%%%%%%%%%%%%%%%%%%%%%%%%%%%%%
%
%
%   This file is included in Registration.tex
%
%   Lablels and section entries are defined in that file.
%
%
%
%%%%%%%%%%%%%%%%%%%%%%%%%%%%%%%%%%%%%%%%%%%%%%%%%%%%%%%%%%%%%%%

For the problem of intra-modality deformable registration, the Insight
toolkit provides an implementation of Thirion's ``demons'' algorithm
\cite{Thirion1995b,Thirion1998}. 
In this implementation, each image is viewed as a set of iso-intensity contours. 
The main idea is that a regular grid of demons forces deform an image by 
pushing the contours in the normal direction. 
The orientation and magnitude of the displacement is 
derived from the instantaneous optical flow equation:

\begin{equation}
\bf{D}(\bf{X}) \cdot \nabla f(\bf{X}) = - \left(m(\bf{X}) - f(\bf{X}) \right)
\label{eqn:OpticalFlow}
\end{equation}

In the above equation, $f(\bf{X})$ is the fixed image and $m(\bf{X})$
is the moving image to be registered and $\bf{D}(\bf{X})$ is the displacement 
or optical flow between the images. It is well known in optical flow
literature that Equation \ref{eqn:OpticalFlow} is insufficient to specify 
$\bf{D}(\bf{X})$ locally and is usually determined using some form of
regularization. For registration, the projection of the vector on the
direction of the intensity gradient is used:

\begin{equation}
\bf{D}(\bf{X}) = - \frac
{\left(  m(\bf{X}) - f(\bf{X}) \right) \nabla f(\bf{X})}
{\left\|  \nabla f \right\|^2 } 
\end{equation}

However, this equation becomes unstable for small values of the image gradient,
resulting in large displacement values. To overcome this problem, Thirion
renormalized the equation such that:

\begin{equation}
\bf{D}(\bf{X}) = - \frac
{\left(  m(\bf{X}) - f(\bf{X}) \right) \nabla f(\bf{X})}
{\left\|  \nabla f \right\|^2 + \left(  m(\bf{X}) - f(\bf{X}) \right)^2 } 
\label{eqn:DemonsDisplacement}
\end{equation}

Starting with an initial deformatin field $\bf{D}^{0}(\bf{X})$, the demons algorithm
iteratively updates the field using Equation \ref{eqn:DemonsDisplacement} such
that the field at the $N$-th iteration is given by:

\begin{equation}
\bf{D}^{N}(\bf{X}) = \bf{D}^{N-1}(\bf{X}) - \frac
{\left(  m(\bf{X}) - f(\bf{X}) \right) \nabla f(\bf{X})}
{\left\|  \nabla f \right\|^2 + \left(  
m(\bf{X}+ \bf{D}^{N-1}(\bf{X}) )
 - f(\bf{X}) \right)^2 } 
\label{eqn:DemonsUpdateEquation}
\end{equation}

Reconstruction of the deformation field is an ill-posed problem where matching the
fixed and moving images have many solutions. For example, since each image pixel
is free to move independently, it is possible that all pixels of one particular
value in $m(\bf{X})$ could map to a single image pixel in $f(\bf{X})$
of the same value. The resulting deformation field may be unrealistic for real-world
applications. An option to solve for the field uniquely is to enforce an 
elastic-like behavior by smoothing the deformation field using a Gaussian filter 
between iterations.

In the toolkit, the demons algorithm is implemented as part of the Finite Difference
Solver (FDS) framework and it use is demostrated in the following example.

\input{DeformableRegistration2.tex} 

