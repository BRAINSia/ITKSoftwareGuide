\chapter{Filtering}

This chapter introduces the most commonly used filters in the toolkit.  Most of
these filters are intended to process images. They will accept one or more
images as input and will produce one or more images as output. Insight is based
on a data pipeline architecture in which the output of one filter is passed as
input to another filter.


\section{Thresholding}
\label{sec:ThresholdingFiltering}

\subsection{Binary Thresholding}
\label{sec:BinaryThresholdingImageFilter}

\input BinaryThresholdImageFilter.tex

\subsection{Thresholding}
\label{sec:ThresholdingImageFilter}

\input ThresholdImageFilter.tex



\section{Casting}
\label{sec:CastingFiltering}

\input CastingImageFilters.tex


\section{Gradients}
\label{sec:GradientFiltering}

Computation of gradients is a fairly common operation in image processing. The
term is sometimes loosely used to refer the gradient vectors or the magnitude
of this gradient. Insight filters attempt to reduce this ambiguity by including
the \emph{magnitude} term when appropriate. Insight provides filters for
computing both the image of gradient vectors and the image of magnitudes.

\subsection{Gradient Magnitude}
\label{sec:GradientMagnitudeImageFilter}

\input GradientMagnitudeImageFilter.tex

\subsection{Gradient Magnitude With Smoothing}
\label{sec:GradientMagnitudeRecursiveGaussianImageFilter}

\input GradientMagnitudeRecursiveGaussianImageFilter.tex




\section{Neighborhood Filters}
\label{sec:NeighborhoodFilters}

The concept locality is frequently encountered in image processing on the form
of filters that compute every output pixel using information from a reduced
region on the neighborhood of the input pixel. The classical form of this
filters are the $3 \times 3$ filters in 2D images. Convolution masks based on
these neighborhoods could perform diverse tasks ranging from noise reduction,
to differential operations, to mathematical morphology.

The Insight toolkit implements an elegant approach for the computation of these
family of filters. The input image is visited by a special iterator called the
\code{SmartNeighborhoodIterator}. This iterator is capable of moving over all
the pixels in an image and for each position it can address the pixels in a
local neighborhood. Operators can be defined in order to specify what
algorithmic operation must be performed on the neighborhood of the input pixel
to compute the value of the output pixel. The following section describes some
of the commonly used filters that take advantage of this construction.  

\subsection{Median Filter}
\label{sec:MedianFilter}

\input MedianImageFilter.tex


\subsection{Mathematical Morphology}
\label{sec:MathematicalMorphology}

\input MathematicalMorphologyFilters.tex



