\chapter{Filtering}

This chapter introduces the most commonly used filters in the toolkit.  Most of
these filters are intended to process images. They will accept one or more
images as input and will produce one or more images as output. Insight is based
on a data pipeline architecture in which the output of one filter is passed as
input to another filter.


\section{Thresholding}
\ifitkFullVersion
\label{sec:ThresholdingFiltering}
\fi

\subsection{Binary Thresholding}
\label{sec:BinaryThresholdingImageFilter}

\ifitkFullVersion
\input{BinaryThresholdImageFilter.tex}
\fi

\subsection{Thresholding}
\label{sec:ThresholdingImageFilter}

\ifitkFullVersion
\input{ThresholdImageFilter.tex}
\fi



\section{Casting}
\label{sec:CastingImageFilters}

The filters discussed in this section perform pixel-wise intensity mappings.
This is usually desired as a parallel action to image pixel-type casting since
the input and output pixel-types have in general different dynamic ranges.

\subsection{Linear Mappings}
\label{sec:IntensityLinearMapping}

\ifitkFullVersion
\input{CastingImageFilters.tex}
\fi

\subsection{Non Linear Mappings}
\label{sec:IntensityNonLinearMapping}

The following filter can be seen as a variant of the casting filters. Its main
difference is the use of a smooth and continuous transtion function of
non-linear form.

\ifitkFullVersion
\input{SigmoidImageFilter.tex}
\fi
  

\section{Gradients}
\label{sec:GradientFiltering}

Computation of gradients is a fairly common operation in image processing. The
term is sometimes loosely used to refer the gradient vectors or the magnitude
of this gradient. Insight filters attempt to reduce this ambiguity by including
the \emph{magnitude} term when appropriate. Insight provides filters for
computing both the image of gradient vectors and the image of magnitudes.

\subsection{Gradient Magnitude}
\label{sec:GradientMagnitudeImageFilter}

\ifitkFullVersion
\input{GradientMagnitudeImageFilter.tex}
\fi

\subsection{Gradient Magnitude With Smoothing}
\label{sec:GradientMagnitudeRecursiveGaussianImageFilter}

\ifitkFullVersion
\input{GradientMagnitudeRecursiveGaussianImageFilter.tex}
\fi


\subsection{Derivative Without Smoothing}
\label{sec:DerivativeImageFilter}

\ifitkFullVersion
\input{DerivativeImageFilter.tex}
\fi




\section{Neighborhood Filters}
\label{sec:NeighborhoodFilters}

The concept locality is frequently encountered in image processing on the form
of filters that compute every output pixel using information from a reduced
region on the neighborhood of the input pixel. The classical form of this
filters are the $3 \times 3$ filters in 2D images. Convolution masks based on
these neighborhoods could perform diverse tasks ranging from noise reduction,
to differential operations, to mathematical morphology.

The Insight toolkit implements an elegant approach for the computation of these
family of filters. The input image is visited by a special iterator called the
\code{NeighborhoodIterator}. This iterator is capable of moving over all
the pixels in an image and for each position it can address the pixels in a
local neighborhood. Operators can be defined in order to specify what
algorithmic operation must be performed on the neighborhood of the input pixel
to compute the value of the output pixel. The following section describes some
of the commonly used filters that take advantage of this construction.  

\subsection{Mean Filter}
\label{sec:MeanFilter}

\ifitkFullVersion
\input{MeanImageFilter.tex}
\fi

\subsection{Median Filter}
\label{sec:MedianFilter}

\ifitkFullVersion
\input{MedianImageFilter.tex}
\fi


\subsection{Mathematical Morphology}
\label{sec:MathematicalMorphology}

Mathematical morphology has proved to be a powerful resource for image
processing and analysis \cite{Serra1982}. Insight implements mathematical
morphology filters using the approach of \code{NeighborhoodIterator}s and
\code{NeighborhoodOperator}s. Two basic flavors of filters are available in the
toolkit, the ones that operate on binary images and the ones that operate on
grayscale images. 

\subsubsection{Binary Filters}
\label{sec:MathematicalMorphologyBinaryFilters}

\ifitkFullVersion
\input{MathematicalMorphologyBinaryFilters.tex}
\fi


\subsubsection{Grayscale Filters}
\label{sec:MathematicalMorphologyGrayscaleFilters}

\ifitkFullVersion
\input{MathematicalMorphologyGrayscaleFilters.tex}
\fi




\section{Smoothing Filters}
\label{sec:SmoothingFilters}

Real image data has a level of uncertainty that is manifested on the
variability of measures assigned to pixels. This is usually interpreted as
noise and considered an undesirable component of the image data. This section
describes several methods that can be applied to reduce noise on images.

\subsection{Blurring}
\label{sec:BlurringFilters}

Blurring is the traditional approach for removing noise from images. It 
usually implemented in the form of a convolution with a kernel. The effect of
this operation on the image spectrum is to attenuate high spatial frequencies.
Different kernels attenuate frequencies in different ways. One of the most
commonly used kernels is the Gaussian. Two implementations of Gaussian
smoothing are available in the toolkit. The first one is based on a traditional
convolution while the other is based on the application of IIR filters that
approximate the convolution with a Gaussian \cite{Deriche1990,Deriche1993}. 

\subsubsection{Discrete Gaussian}
\label{sec:DiscreteGaussianImageFilter}

\ifitkFullVersion
\input{DiscreteGaussianImageFilter.tex}
\fi


\subsubsection{Binomial Blurring}
\label{sec:BinomialBlurImageFilter}

\ifitkFullVersion
\input{BinomialBlurImageFilter.tex}
\fi

\subsubsection{Recursive Gaussian IIR}
\label{sec:RecursiveGaussianImageFilter}

\ifitkFullVersion
\input{SmoothingRecursiveGaussianImageFilter.tex}
\fi



\subsection{Edge Preserving Smoothing}
\label{sec:EdgePreservingSmoothingFilters}

\subsubsection{Introduction to Anisotropic Diffusion}
\label{sec:IntroductionAnisotropicDiffusion}
\ifitkFullVersion
%
%
%  This file in inserted in the Filtering.tex file.
%
%

The drawback of image denoising (smoothing) is that it tends to blur away the
sharp boundaries in the image that help to distinguish between the
larger-scale anatomical structures that one is trying to characterize (which
also limits the size of the smoothing kernels in most applications).  Even in
cases where smoothing does not obliterate boundaries, it tends to distort the
fine structure of the image and thereby changes subtle aspects of the
anatomical shapes in question.

Perona and Malik \cite{Perona1990} introduced an alternative to
linear-filtering that they called \emph{anisotropic diffusion}.  Anisotropic
diffusion is closely related to the earlier work of Grossberg
\cite{Grossberg1984}, who used similar nonlinear diffusion processes to model
human vision.  The motivation for anisotropic diffusion (also called
\emph{nonuniform} or \emph{variable conductance} diffusion) is that a Gaussian
smoothed image is a single time slice of the solution to the heat equation,
that has the original image as its initial conditions.  Thus, the solution to
\begin{equation} \frac{\partial g(x, y, t) }{\partial t} = \nabla \cdot \nabla
g(x, y, t), \end{equation} where $g(x, y, 0) = f(x, y)$ is the input image, is
$g(x, y, t) = G(\sqrt{2t}) \otimes f(x, y)$, where $G(\sigma)$ is a Gaussian
with standard deviation $\sigma$.

Anisotropic diffusion includes a variable conductance term that, in turn,
depends on the differential structure of the image.  Thus, the variable
conductance can be formulated to limit the smoothing at ``edges'' in images, as
measured by high gradient magnitude, for example. \begin{equation} g_{t} = \nabla \cdot
c(\left| \nabla g \right|) \nabla g, \label{eq:aniso} \end{equation} where, for
notational convenience, we leave off the independent parameters of $g$ and use
the subscripts with respect to those parameters to indicate partial
derivatives.  The function $c(|\nabla g|)$ is a fuzzy cutoff that reduces the
conductance at areas of large $|\nabla g|$, and can be any one of a number of
functions.  The literature has shown \begin{equation} c(|\nabla g|) =
e^{-\frac{|\nabla g|^{2}}{2k^{2}}} \end{equation} to be quite effective.
Notice that conductance term introduces a free parameter $k$, the {\em
conductance parameter}, that controls the sensitivity of the process to edge
contrast.  Thus, anisotropic diffusion entails two free parameters: the
conductance parameter, $k$, and the time parameter, $t$, that is analogous to
$\sigma$, the effective width of the filter when using Gaussian kernels.

Equation \ref{eq:aniso} is a nonlinear partial differential equation that can
be solved on a discrete grid using finite forward differences.  Thus, the
smoothed image is obtained only by an iterative process, not a convolution or
non-stationary, linear filter.  Typically, the number of iterations required
for practical results are small, and large 2D images can be processed in
several tens of seconds using carefully written code running on modern, general
purpose, single-processor computers.  The technique applies readily and
effectively to 3D images, but requires more processing time.

In the early 1990's several research groups \cite{Gerig1991,Whitaker1993d}
demonstrated the effectiveness of anisotropic diffusion on medical images.  In
a series of papers on the subject
\cite{Whitaker1993,Whitaker1993b,Whitaker1993c,Whitaker1993d,Whitaker-thesis,Whitaker1994},
Whitaker described a detailed analytical and empirical analysis, introduced a
smoothing term in the conductance that made the process more robust, invented a
numerical scheme that virtually eliminated directional artifacts in the
original algorithm, and generalized anisotropic diffusion to vector-valued
images, an image processing technique that can be used on vector-valued medical
data (such as the color cryosection data of the Visible Human Project).

For a vector-valued input $\vec{F}:U \mapsto \Re^{m}$ the process takes the
form \begin{equation} \vec{F}_{t} = \nabla \cdot c({\cal D}\vec{F}) \vec{F},
\label{eq:vector_diff} \end{equation} where ${\cal D}\vec{F}$ is a {\em
dissimilarity} measure of $\vec{F}$, a generalization of the gradient magnitude
to vector-valued images, that can incorporate linear and nonlinear coordinate
transformations on the range of $\vec{F}$.  In this way, the smoothing of the
multiple images associated with vector-valued data is coupled through the
conductance term, that fuses the information in the different images.  Thus
vector-valued, nonlinear diffusion can combine low-level image features (e.g.
edges) across all ``channels'' of a vector-valued image in order to preserve or
enhance those features in all of image ``channels''.

Vector-valued anisotropic diffusion is useful for denoising data from devices
that produce multiple values such as MRI or color photography.  When performing
nonlinear diffusion on a color image, the color channels are diffused
separately, but linked through the conductance term. Vector-valued diffusion
is also useful for processing registered data from different devices or for
denoising higher-order geometric or statistical features from scalar-valued
images \cite{Whitaker1994,Yoo1993}.

The output of anisotropic diffusion is an image or set of images that
demonstrates reduced noise and texture but preserves, and can also enhance,
edges.  Such images are useful for a variety of  processes including
statistical classification, visualization, and geometric feature extraction.
Previous work has shown \cite{Whitaker-thesis} that anisotropic diffusion, over
a wide range of conductance parameters, offers quantifiable advantages over
linear filtering for edge detection in medical images.

Since the effectiveness of nonlinear diffusion was first demonstrated, numerous
variations of this approach have surfaced in the literature \cite{Romeny1994}.
These include alternatives for constructing dissimilarity measures
\cite{Sapiro1996}, directional (i.e., tensor-valued) conductance terms
\cite{Weickert1996,Alvarez1994} and level set interpretations
\cite{Whitaker2001}.

\fi


\subsubsection{Gradient Anisotropic Diffusion}
\label{sec:GradientAnisotropicDiffusionImageFilter}

\ifitkFullVersion
\input{GradientAnisotropicDiffusionImageFilter.tex}
\fi



\subsubsection{Curvature Anisotropic Diffusion}
\label{sec:CurvatureAnisotropicDiffusionImageFilter}

\ifitkFullVersion
\input{CurvatureAnisotropicDiffusionImageFilter.tex}
\fi

\subsubsection{Curvature Flow}
\label{sec:CurvatureFlowImageFilter}

\ifitkFullVersion
\input{CurvatureFlowImageFilter.tex}
\fi

\subsubsection{MinMaxCurvature Flow}
\label{sec:MinMaxCurvatureFlowImageFilter}

\ifitkFullVersion
\input{MinMaxCurvatureFlowImageFilter.tex}
\fi


\subsubsection{Bilateral Filter}
\label{sec:BilateralImageFilter}

\ifitkFullVersion
\input{BilateralImageFilter.tex}
\fi



\subsection{Edge Preserving Smoothing in Vector Images}
\label{sec:VectorAnisotropicDiffusion}

Anisotropic diffusion can also be applied to images whose pixels are vectors.
In this case the diffusion is computed independently for each vector component.
The following classes implement versions of anisotropic diffusion on vector images.


\subsubsection{Vector Gradient Anisotropic Diffusion}
\label{sec:VectorGradientAnisotropicDiffusionImageFilter}

\ifitkFullVersion
\input{VectorGradientAnisotropicDiffusionImageFilter.tex}
\fi

\subsubsection{Vector Curvature Anisotropic Diffusion}
\label{sec:VectorCurvatureAnisotropicDiffusionImageFilter}

\ifitkFullVersion
\input{VectorCurvatureAnisotropicDiffusionImageFilter.tex}
\fi



\subsection{Edge Preserving Smoothing in Color Images}
\label{sec:ColorAnisotropicDiffusion}

\subsubsection{Gradient Anisotropic Diffusion}
\label{sec:ColorGradientAnisotropicDiffusion}

\ifitkFullVersion
\input{RGBGradientAnisotropicDiffusionImageFilter.tex}
\fi

\subsubsection{Curvature Anisotropic Diffusion}
\label{sec:ColorCurvatureAnisotropicDiffusion}

\ifitkFullVersion
\input{RGBCurvatureAnisotropicDiffusionImageFilter.tex}
\fi



\section{Distance Map}
\label{sec:DistanceMap}

\ifitkFullVersion
\input{DanielssonDistanceMapImageFilter.tex}
\fi



\section{Geometrical Transformations}
\label{sec:GeometricalTransformationFilters}

\subsection{Resample Image Filter}
\label{sec:ResampleImageFilter}

\subsubsection{Introduction}

\ifitkFullVersion
\input{ResampleImageFilter.tex}
\fi

\subsubsection{Importance of Spacing and Origin}
\ifitkFullVersion
\input{ResampleImageFilter2.tex}
\fi

\subsubsection{A full example}
\ifitkFullVersion
\input{ResampleImageFilter3.tex}
\fi

\subsubsection{Rotating an Image}
\ifitkFullVersion
\input{ResampleImageFilter4.tex}
\fi

\subsubsection{Rotating and Scaling an Image}
\ifitkFullVersion
\input{ResampleImageFilter5.tex}
\fi


