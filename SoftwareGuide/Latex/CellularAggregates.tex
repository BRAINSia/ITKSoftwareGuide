
\chapter{Cellular Aggregates}

This chapter introduces the use of Cellular Aggregates for performing image
analysis tasks. Cellular Aggregates are an abtraction and simplification of
groups of biological cells. The emulation of cellular behavior is used for
taking advantage of their natural capabilities for generating shapes, exploring
space and cooperating for solving complex situations. 

The main application of Cellular Aggregates in the Insight Toolkit is image
segmentation. The following sections present the basic concepts used in ITK for
implementing Cellular Aggregates.

\section{Overview}

In an over-simplified model of a biological cell, the following elements can be
considered to be the fundamental control system of the cell.

\begin{itemize}
\item Genes
\item Proteins
\end{itemize}

\section{Gene Network Modeling}

In order to model cell behavior it is necessary to perform simulations of their 
internal gene network evolution. A Gene Network is simplified here in the with the
following model

A gene has several regions, namely

a coding region
a control region

The control region is composed of regulatory domains. Each domain is a section
of DNA with enough specific characteristics to make it identifiable among all
other domains.  Regulatory domains can be promoters, enhancers, silencers or
represors.  In the ITK model, the categories have been reduced to just
enhancers and represors.  Proteins in the cell, have domains that exibith
certain level of affinity for the regulatory domains of the gene. The affinity
of a particular protein domain for a regulatory domain of a gene will define
the probability of a molecule of the protein to be bound to the regulatory
domain for a certain period of time.  During this time, the regulatory domain
will influence the expression of the gene. Since we simplified the categories
of regulatory domains to just \emph{enhancers} and \emph{represors}, the
actions of the regulatory domains can be represented by the following generic
expression

\begin{equation}
\frac{\partial{G}}{\partial t} = \left[ ABCDE + A\over{B} \right]
\end{equation}

Where the $\over{B}$ represent the repressors and the normal letters represents
the enhancers. It is known from boolean algebra that any boolean polynomial can
be expressed as a sum of products composed by the polynomial terms and their
negations.

The values to be assigned to the letters are the probabilities of this
regulatory domains to be bound by a protein at a particular time. This probability
is estimated as the weighted sum of protein concentrations. The weights are the
affinities of each of the protein domains for the particular regulatory domain.

For example, for the regulatory domain $A$, we compute the probability o a
proteing to be bound on this domain as the sum over all the proteins, of
$A$-complementary domains affinities by the protein concentration.


\begin{equation}
A = \sum_i P_i \cdot \sum_j D  \cdot Aff
\end{equation}

Where Aff is the affinity of the Domain $j-th$ of the $i-th$ protein for the
$A$ regulatory domain. By considering only one-to-one domains, that is, a domain
of a protein has affinity only for a single type of regulatory domain. The affinity
may be of any value in the range [0,1] but it will be limited to a single domain.

This is an ad hoc limitation of this model, introduced only with the goal of simplifying
the mathematical construction and increasing computational performance.


A review on cell regulation through protein interaction domains is made by
Pawson and Nash in \cite{Pawson2003}.


