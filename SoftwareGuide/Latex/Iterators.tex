\chapter{Iterators}

% Questions: do I talk about issues like image regions?  i.e. which regions to
%  always work with?  Do I talk about issues with the neighborhoods such as how
% to do convolution and other neighborhood framework objects ?

%LOOKS LIKE WITH INDEX ITERATOR IS NOT NEEDED IN BINARY FUNCTOR IMAGE FILTER

\section(Introduction)
\label{sec:IteratorsIntroduction}

This chapter introduces image iterators.  Image iterators are an important and
powerful tool for image processing using ITK.  They are the preferred way to
work with image pixels, can speed up algorithms, and help to generalize many
common image operations to arbitrary dimensionality.

An iterator is a generalization of a memory pointer.  Generic
programming models decompose programming space into functionally orthogonal principle
components, including algorithms and data containers.  Iterators provide the glue
between algorithms and data containers.  They allow  algorithms to access values stored in
data containers of any type that uses a common interface. [enable code reuse,
generalize to higher dimensions]  iterator steps through the container one
element at a time.

[give an example of just a real-world, normal iterator]
The simplest example of an iterator is a wrapper for a C memory pointer.  
[check stroustrup for info here and possible example, also reference for STL?]
[simple example of how a generic iterator, for 3 dimensions show a for loop and
image access contrasted with a single image]


[describe a standard ITK interface]
Most ITK iterators define the following methods. [a bulleted definition list?]
        GoToBegin()
        GoToEnd()
        IsAtEnd()
        IsAtBegin()
        Operator++
        Opterator--

[where traditional iterators overload the * operator, in ITK we have chosen to
use Set Get]
        Get
        Set
Set methods are only defined in const iterators.
Some iterators also supply a Value() method to allow dereferencing and modifying a
pixel value in a single step.  For example: [give example of how if you wanted
to do something like *it++ you would have to write it as it.Set(it.Get() + 1),
whereas it.Value()++ saves one dereference]

[image adaptors]
The use of Set, Get also supports image adaptors (section reference?)

[three basic types of image iterators: single pixel access methods, single
pixel methods which walk special paths, neighborhood access methods,]

\section{Image Iterators}
\label{sec:ImageIterators}



\subsection{itk::ImageRegionIterator}
\label{sec:itkImageRegionIterator}
./Common/itkImageRegionConstIterator.h
./Common/itkImageRegionIterator.h
(./Common/itkImageConstIterator.h base class)
(./Common/itkImageIterator.h base class)

Describe typical use cases.

Describe any special peculiarities of the interface and the guarantees of
complexity, etc.  Guarantees: will walk a region in same order on two different
images, no guarantees of what that order will be necessarily.

Example code

\subsection{itk::ImageRegionIteratorWithIndex}
\label{sec:itkImageRegionIteratorWithIndex}
./Common/itkImageRegionConstIteratorWithIndex.h
./Common/itkImageRegionIteratorWithIndex.h
(./Common/itkImageConstIteratorWithIndex.h  base class )
(./Common/itkImageIteratorWithIndex.h  base class )

\subsection{itk::ImageSliceIteratorWithIndex}
\label{sec:itkImageSliceIteratorWithIndex}
./Common/itkImageSliceConstIteratorWithIndex.
./Common/itkImageSliceIteratorWithIndex.h

\subsection{itk::ImageLinearIteratorWithIndex}
\label{sec:itkImageLinearIteratorWithIndex}
./Common/itkImageLinearConstIteratorWithIndex.h
./Common/itkImageLinearIteratorWithIndex.h

\subsection{itk::ImageRandomConstIteratorWithIndex}
\label{sec:itkImageRandomConstIteratorWithIndex}
./Common/itkImageRandomConstIteratorWithIndex.h
./Common/itkImageRandomIteratorWithIndex.h


\section{Conditional Iterators}
\label{sec:ConditionalIterators}

[Introduction and overview]

./Common/itkConditionalConstIterator.h (BaseClass)
./Common/itkConditionalIterator.h (BaseClass)
./Common/itkFloodFilledFunctionConditionalConstIterator.h (BaseClass)
./Common/itkFloodFilledFunctionConditionalIterator.h (BaseClass)


%[ here are all classes where these filters are used:
% ./BasicFilters/itkConfidenceConnectedImageFilter.txx (ImageFunction)
% ./BasicFilters/itkConnectedThresholdImageFilter.txx (ImageFunction)
% ./BasicFilters/itkIsolatedConnectedImageFilter.txx (ImageFunction)
% ./BasicFilters/itkNeighborhoodConnectedImageFilter.txx (ImageFunction)
%
% ./Common/itkBinaryBallStructuringElement.txx (SpatialFunction)
% ./Common/itkBloxCoreAtomImage.txx (SpatialFunction)
% ./BasicFilters/itkBloxBoundaryPointToCoreAtomImageFilter.txx (SpatialFunction)
% ./BasicFilters/itkBloxBoundaryPointImageToBloxBoundaryProfileImageFilter.txx (SpatialFunction)
%]

\subsection{itk::FloodFilledImageFunctionConditionalIterator}
\label{itk::FloodFilledImageFunctionConditionalIterator}
./Common/itkFloodFilledImageFunctionConditionalConstIterator.h
./Common/itkFloodFilledImageFunctionConditionalIterator.h


\subsection{itk::FloodFilledSpatialFunctionConditionalIterator}
\label{itk::FloodFilledSpatialFunctionConditionalIterator}
./Common/itkFloodFilledSpatialFunctionConditionalConstIterator.h
./Common/itkFloodFilledSpatialFunctionConditionalIterator.h


\section{Neighborhood Iterators}
\label{sec:NeighborhoodIterators}

[Introduction goes here]
[Be sure to reference Section: Neighborhood Filters]

% Example: derivative
% Example: convolution filtering
% Example: boundary conditions
% Example: walking faces

\subsection{itk::NeighborhoodIterator}
\label{sec:itkNeighborhoodIterator}
./Common/itkConstNeighborhoodIterator.h
./Common/itkNeighborhoodIterator.h

\subsection{itk::ShapedNeighborhoodIterator}
\label{sec:itkShapedNeighborhoodIterator}
./Common/itkConstShapedNeighborhoodIterator.h
./Common/itkShapedNeighborhoodIterator.h






















