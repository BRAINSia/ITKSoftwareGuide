\chapter*{Abstract}
\noindent
The Insight Toolkit \href{http://www.itk.org}{(ITK)} is an open-source software
toolkit for performing registration and segmentation. Segmentation is the
process of identifying and classifying data found in a digitally sampled
representation. Typically the sampled representation is an image acquired from
such medical instrumentation as CT or MRI scanners. Registration is the task of
aligning or developing correspondences between data. For example, in the
medical environment, a CT scan may be aligned with a MRI scan in order to
combine the information contained in both.

ITK is implemented in C++. ITK is cross-platform, using the 
\href{http://www.cmake.org}{CMake} 
build environment to manage the compilation process. In addition, an automated
wrapping process generates interfaces between C++ and interpreted programming
languages such as Tcl, Java, and \href{http://www.python.org}{Python}
\href{http://public.kitware.com/Cable/HTML/Index.html}{(Cable)}. This enables
developers to create software using a variety of programming languages. ITK's
C++ implementation style is referred to as generic programming (i.e., using
templated code). Such C++ templating means that the code is highly efficient, 
and that many software problems are discovered at compile-time, rather than at 
run-time during program execution.

Because ITK is an open-source project, developers from around the world can
use, debug, maintain, and extend the software. ITK uses a model of software
development referred to as Extreme Programming. Extreme Programming collapses
the usual software creation methodology into a simultaneous and iterative
process of design-implement-test-release. The key features of Extreme
Programming are communication and testing. Communication among the members of
the ITK community is what helps manage the rapid evolution of the software.
Testing is what keeps the software stable. In ITK, an extensive testing process
(using 
\href{http://public.kitware.com/dashboard.php}{Dart}) is in place that 
measures the quality on a daily basis. The ITK
Testing Dashboard is posted continuously reflecting the quality of the
software.

This book is a guide to using and developing with ITK. The
\href{http://www.itk.org/cgi-bin/cvsweb.cgi/Insight/Examples/?cvsroot=Insight}{Insight/Examples}
directory and sample code provides a companion to the material presented here.
The most recent version of this document is available on-line at
\url{http://www.itk.org/ItkSoftwareGuide.pdf}.



