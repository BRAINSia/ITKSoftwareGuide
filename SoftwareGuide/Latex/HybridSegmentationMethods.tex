%%%%%%%%%%%%%%%%%%%%%%%%%%%%%%%%%%%%%%%%%%%%%%%%%%%%%%%%%%%%%%%%%%%%%%%%
%
%
%     This file is included from the file   Segmentation.tex
% 
%     Section tag and label are placed in this top file.
%
%
%
%%%%%%%%%%%%%%%%%%%%%%%%%%%%%%%%%%%%%%%%%%%%%%%%%%%%%%%%%%%%%%%%%%%%%%%%

\subsection{Introduction}
\label{sec:HybridSegmentationIntroduction}


It is sometimes convenient to combine several segmentation strategies with the
aim of taking advantage of their qualities an compensate their vulnerabilities.
The synergy between fundamentally different methodologies tends to result in
robustness and higher segmentation quality.  This section illustrates an hybrid
approach for segmentation in which two different strategies are configured to
work together. In this case, an input image is first processed by a filter
based on the concept of region growing. The criterion of acceptance in the
region is defined by a similarity measure that evaluates how homogeneous is the
path between to pixels. The output of this filter is used as a prior for
another filter that performs a full partition of the image space and then work
joining and splitting regions in order to optimize an homogeneity measure.
Details on the concepts behind those methods have been discussed in the
litterature
\cite{Angelini2002,Udupa2002,Jin2002,Imielinska2001,Imielinska2000a,Imielinska2000b}



\subsection{Background}
\label{sec:HybridSegmentationBackground}


%%%%%%%%%%%%%%%%%%%%%%%%%%%%%%%%%%%%%%%%%%%%%%%%%%%%%%%%%%%%%%%%%
%
%  Here is an example of how to include an image in a figure
%
%  The file HybridSegmentationOutput1.jpg should be in the "Art"
%  directory. CMake will convert it to EPS before running latex. 
%
%%%%%%%%%%%%%%%%%%%%%%%%%%%%%%%%%%%%%%%%%%%%%%%%%%%%%%%%%%%%%%%%%

\begin{figure}
\center
\includegraphics[width=12cm]{HybridSegmentationOutput1.eps}
\caption{This is the image caption...}
\label{fig:HybridSegmentationImage1}
\end{figure}

The Figure \ref{fig:HybridSegmentationImage1} present the results
of the hybrid segmentation ...




%%%%%%%%%%%%%%%%%%%%%%%%%%%%%%%%%%%%%%%%%%%%%%%%%%%%%%%%%%%%%%%%%
%
%  Here is an example of how to include diagram in a figure
%
%  The file HybridSegmentationDiagram1.fig should be in the "Art"
%  directory. CMake will convert it to EPS before running latex. 
%
%%%%%%%%%%%%%%%%%%%%%%%%%%%%%%%%%%%%%%%%%%%%%%%%%%%%%%%%%%%%%%%%%

\begin{figure}
\center
\includegraphics[width=12cm]{HybridSegmentationDiagram1.eps}
\caption{This is the diagram caption...}
\label{fig:HybridSegmentationDiagram1}
\end{figure}

The Figure \ref{fig:HybridSegmentationDiagram1} illustrates the main
components of the hybrid segmentation algorithm...




%%%%%%%%%%%%%%%%%%%%%%%%%%%%%%%%%%%%%%%%%%%%%%%%%%%%%%%%%%%%%%%%%
%
%  Here is an example of how to include equations
%
%%%%%%%%%%%%%%%%%%%%%%%%%%%%%%%%%%%%%%%%%%%%%%%%%%%%%%%%%%%%%%%%%


\begin{equation}
MS(A,B) = \frac{1}{N} \sum_i^N \left( A_i - B_i \right)^2
\end{equation}





\subsection{Example}
\label{sec:HybridSegmentationExample1}

\input{HybridSegmentationFuzzyVoronoi.tex}


