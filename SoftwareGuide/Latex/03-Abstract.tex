\chapter*{Abstract}
\noindent
The Insight Toolkit \href{http://www.itk.org}{(ITK)} is an open-source
software toolkit for performing registration and
segmentation. \emph{Segmentation} is the process of identifying and
classifying data found in a digitally sampled
representation. Typically the sampled representation is an image
acquired from such medical instrumentation as CT or MRI
scanners. \emph{Registration} is the task of aligning or developing
correspondences between data. For example, in the medical environment,
a CT scan may be aligned with a MRI scan in order to combine the
information contained in both.

ITK is implemented in C++. It is cross-platform, using a build
environment known as \href{http://www.cmake.org}{CMake} to manage the
compilation process in a platform-independent way.
In addition, an automated wrapping process implemented in python
generates interfaces between C++ and interpreted programming languages
such as
\href{http://www.python.org}{Python}. This
\href{http://tcl.sourceforge.net}{Tcl},
\href{http://java.sun.com}{Java}, and
enables developers to create software using a variety of programming
languages. ITK's C++ implementation style is referred to as generic
programming, which is to say that it uses templates so that the same
code can be applied \emph{generically} to any class or type that
happens to support the operations used. Such C++
templating means that the code is highly efficient, and that many
software problems are discovered at compile-time, rather than at
run-time during program execution.

In addition to the automated wrapping, the
\href{http://www.itk.org/Wiki/SimpleITK}{Simple ITK} project provides
a streamlined interface to ITK that is available for Python, Java,
CSharp, R, Tcl and Ruby. 

Because ITK is an open-source project, developers from around the
world can use, debug, maintain, and extend the software. ITK uses a
model of software development referred to as Extreme
Programming. Extreme Programming collapses the usual software creation
methodology into a simultaneous and iterative process of
design-implement-test-release. The key features of Extreme Programming
are communication and testing. Communication among the members of the
ITK community is what helps manage the rapid evolution of the
software.  Testing is what keeps the software stable. In ITK, an
extensive testing process (using a system known as
\href{http://public.kitware.com/dashboard.php?name=itk}{CDash}) is in place that
measures the quality on a daily basis. The ITK Testing Dashboard is
posted continuously, reflecting the quality of the software at any
moment.

This book is a guide to using and developing with ITK. The sample code
in the
\href{http://itk.org/gitweb?p=ITK.git}
{directory} provides a companion to the material presented here.  The
most recent version of this document is available online at
\url{http://www.itk.org/ItkSoftwareGuide.pdf}.


