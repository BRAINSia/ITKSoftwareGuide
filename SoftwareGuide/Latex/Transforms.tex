In the toolkit, \code{itk::Transform} objects encapsulates the mapping of
points, vectors and covariant vectors from an input space to an output space.
The distinction between points, vectors and covariant vectors has already
been discussed in Chapter \ref{sec:DataRepresentation}. If a transform is
invertible, back transform methods are also provided. Currently, 
ITK provide a variety of transfroms from simple translation, rotation and 
scaling to general affine and kernel transforms. Note that, although we
discuss transforms in context registration in this section, transforms
are general and can used for other applications.

Typically, for each transform type several methods are provided for setting
the parameters. For example, \code{Euler2DTransform} provide methods for
separatedly setting the offset, the angle, and the entire rotation matrix.
However, for use in the registration framework, the parameters must also
be represented by a flat \code{Array<double>} to allow communication
with generic optimizers. In the case of \code{Euler2DTransform}, the transform
is also defined by three doubles: the first representing the angle and 
the last two the offset.

%Talk about jacobian
%Go through each transform
