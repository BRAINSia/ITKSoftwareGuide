
\def\tableconfiguration{ | p{4cm} | p{3cm} | p{3cm} | p{4cm} | }

\index{itk::Transform|textbf}

In the toolkit, \code{itk::Transform} objects encapsulates the mapping of
points and vectors from an input space to an output space.
If a transform is invertible, back transform methods are also provided. 
Currently, ITK provide a variety of transforms from simple translation, 
rotation and scaling to general affine and kernel transforms. 
Note that, although we discuss transforms in context registration in this section, 
transforms are general and can used for other applications. Some of the commonly used 
transforms will be discussed in detail later in this section.

\index{itk::Transform!TransformPoint()}
\index{itk::Transform!TransformVector()}
\index{itk::Transform!TransformCovariantVector()}
Points, vectors and covariant vectors are transformed using methods
\code{TransformPoint()}, \code{TransformVector()} and \code{TransformCovariantVector()}.
The distinction between points, vectors and covariant vectors has already
been discussed in Chapter \ref{sec:DataRepresentation}. 

\index{itk::Transform!SetParameters()}
Typically each transform type have several methods for setting
the parameters. For example, \code{Euler2DTransform} provide methods for
separately setting the offset, the angle, and the entire rotation matrix.
However, for use in the registration framework, the parameters must also
be represented by a flat \code{Array<double>} to allow communication
with generic optimizers. In the case of \code{Euler2DTransform}, the transform
is also defined by three doubles: the first representing the angle and 
the last two the offset. The flat array of parameters is defined 
using \code{SetParameters()}. A description of the parameters and their
ordering is documented in the header file of each transform. 

\index{itk::Transform!GetJacobian()}
Another requirement of the registration framework is the computation of the
transformation Jacobian. In general, metrics require the knowledge of 
the Jacobian in order to compute the metric derivatives. 
The Jacobian is a matrix whose element are the partial derivatives of the 
output point with respect to the array of parameters that defines the 
transform:

\begin{equation}
J=\left[ \begin{array}{cccc}
\frac{\partial x_{1}}{\partial p_{1}} & 
\frac{\partial x_{1}}{\partial p_{2}} & 
\cdots  & \frac{\partial x_{1}}{\partial p_{m}}\\
\frac{\partial x_{2}}{\partial p_{1}} & 
\frac{\partial x_{2}}{\partial p_{2}} & 
\cdots  & \frac{\partial x_{2}}{\partial p_{m}}\\
\vdots  & \vdots  & \ddots  & \vdots \\
\frac{\partial x_{n}}{\partial p_{1}} & 
\frac{\partial x_{n}}{\partial p_{2}} & 
\cdots  & \frac{\partial x_{n}}{\partial p_{m}}
\end{array}\right]
\end{equation}

Within this framework, the Jacobian is represented by a \code{Array2D<double>}
and is obtained from the transform by method \code{GetJacobian()}


\subsection{Identity Transform}
\label{sec:IdentityTransform}

\begin{center}
\begin{tabular}{\tableconfiguration}
\hline
\textbf{Behavior} &
\textbf{Number of parameters} &
\textbf{Parameter Ordering} &
\textbf{Restrictions} \\
\hline\hline
Maps every point to itself, every vector to itself and every covariant vector to itself.  & 
0 &
  &  
Only defined when the input and output space has the same number of dimensions. \\
\hline
\end{tabular}
\end{center}





\subsection{Translation Transform}
\label{sec:TranslationTransform}

\begin{center}
\begin{tabular}{\tableconfiguration}
\hline
\textbf{Behavior} &
\textbf{Number of parameters} &
\textbf{Parameter Ordering} &
\textbf{Restrictions} \\
\hline\hline
Represents a simple translation of points in the input space
and has no effect on vectors or covariant vectors. &
Same as the input space dimension. &
The i-th parameter represents the translation in the i-th dimension. &
Only defined when the input and output space has the same number of dimensions. \\
\hline
\end{tabular}
\end{center}





\subsection{Scale Transform}
\label{sec:ScaleTransform}

\begin{center}
\begin{tabular}{\tableconfiguration}
\hline
\textbf{Behavior} &
\textbf{Number of parameters} &
\textbf{Parameter Ordering} &
\textbf{Restrictions} \\
\hline\hline
Represents a simple scaling of the vector space. Each component of a point, vector
or covariant vector is multiplied by the user defined scaling factor. &
Same as the input space dimension. &
The i-th parameter represents the scaling in the i-th dimension. &
Only defined when the input and output space has the same number of dimensions. \\
\hline
\end{tabular}
\end{center}





\subsection{Euler2DTransform}
\label{sec:Euler2DTransform}

\begin{center}
\begin{tabular}{\tableconfiguration}
\hline
\textbf{Behavior} &
\textbf{Number of parameters} &
\textbf{Parameter Ordering} &
\textbf{Restrictions} \\
\hline\hline
Represents a 2D rotation and a 2D translation. Note that the translation
component has no effect on the transformation of vectors and covariant vectors. &
3 &
The first parameter is the angle in radian and the last two parameters
are the translation in each dimension. &
Only defined for two-dimensional input and output spaces. \\
\hline
\end{tabular}
\end{center}





\subsection{Similarity2DTransform}
\label{sec:Similarity2DTransform}

\begin{center}
\begin{tabular}{\tableconfiguration}
\hline
\textbf{Behavior} &
\textbf{Number of parameters} &
\textbf{Parameter Ordering} &
\textbf{Restrictions} \\
\hline\hline
Represents a 2D rotation, homogeneous scaling and a 2D translation. Note that
the translation component has no effect on the transformation of vectors and
covariant vectors. & 
4 &
The first parameter is the angle in radian, the second the scaling factor for
all dimensions and the last two parameters are the translation in each
dimension. & 
Only defined for two-dimensional input and output spaces. \\
\hline
\end{tabular}
\end{center}


\subsection{QuaternionRigidTransform}
\label{sec:QuaternionRigidTransform}

\begin{center}
\begin{tabular}{\tableconfiguration}
\hline
\textbf{Behavior} &
\textbf{Number of parameters} &
\textbf{Parameter Ordering} &
\textbf{Restrictions} \\
\hline\hline
Represents a 3D rotation and a 3D translation. The rotation is specified as a
quaternion, defined by a vector of four numbers $\bf{q}$.  The relationship
between quaternion and rotation about vector $\bf{n}$ by angle $\theta$ is as
follows: \[ \bf{q} = (\bf{n}\sin(\theta/2), \cos(\theta/2))\] Note that if the
quaternion is not of unit length, scaling will also result. &
7 &
The first four parameters defines the quaternion and the last three parameters
the translation in each dimension. &
Only defined for three-dimensional input and output spaces. \\
\hline
\end{tabular}
\end{center}



\subsection{VersorRigid3DTransform}
\label{sec:VersorRigid3DTransform}


\begin{center}
\begin{tabular}{\tableconfiguration}
\hline
\textbf{Behavior} &
\textbf{Number of parameters} &
\textbf{Parameter Ordering} &
\textbf{Restrictions} \\
\hline\hline
Represents a 3D rotation and a 3D translation. The rotation is specified a
versor or unit quaternion, defined by a vector of three numbers.
These three numbers corresponds to the first three components of a quaternion.
The fourth component of the quaternion is derived such that the quaternion is of
unit length. &
6 &
The first three parameters defines the versor and the last three parameters the
translation in each dimension. &
Only defined for three-dimensional input and output spaces. \\
\hline
\end{tabular}
\end{center}



\subsection{AffineTransform}
\label{sec:AffineTransform}

\begin{center}
\begin{tabular}{\tableconfiguration}
\hline
\textbf{Behavior} &
\textbf{Number of parameters} &
\textbf{Parameter Ordering} &
\textbf{Restrictions} \\
\hline\hline
Represents an affine transform composed of rotation, scaling, shearing and
translation. The transform is specified by a $N \times N$ matrix and a $N
\times 1$ vector where $N$ is space dimension. &
$(N+1) \times N$ &
The first $N \times N$ parameters defines the matrix in column-major order
(where the column index varies the fastest).  The last $N$ parameters defines
the translate for each dimension. &
Only defined when the input and output space have the same dimension. \\
\hline
\end{tabular}
\end{center}





