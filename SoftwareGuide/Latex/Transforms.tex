In the toolkit, \code{itk::Transform} objects encapsulates the mapping of
points, vectors and covariant vectors from an input space to an output space.
The distinction between points, vectors and covariant vectors has already
been discussed in Chapter \ref{sec:DataRepresentation}. If a transform is
invertible, back transform methods are also provided. Currently, 
ITK provide a variety of transfroms from simple translation, rotation and 
scaling to general affine and kernel transforms. Note that, although we
discuss transforms in context registration in this section, transforms
are general and can used for other applications. Some of the commonly used 
transforms will be discussed in detail later in this section.

Typically, each transform type several methods are provided for setting
the parameters. For example, \code{Euler2DTransform} provide methods for
separatedly setting the offset, the angle, and the entire rotation matrix.
However, for use in the registration framework, the parameters must also
be represented by a flat \code{Array<double>} to allow communication
with generic optimizers. In the case of \code{Euler2DTransform}, the transform
is also defined by three doubles: the first representing the angle and 
the last two the offset. The a description of the parameters and their
ordering is documented in the header file of each transform.

Another requirement of the registration framework is the of the
transformation Jacobian. In general, metrics require the knowledge of 
the Jacobian in order to compute the metric derivatives. 
The Jacobian is a matrix whose element are the partial derivatives of the 
output point with respect to the array of parameters that defines the 
transform:

\begin{equation}
J=\left[ \begin{array}{cccc}
\frac{\partial x_{1}}{\partial p_{1}} & 
\frac{\partial x_{2}}{\partial p_{1}} & 
\cdots  & \frac{\partial x_{n}}{\partial p_{1}}\\
\frac{\partial x_{1}}{\partial p_{2}} & 
\frac{\partial x_{2}}{\partial p_{2}} & 
\cdots  & \frac{\partial x_{n}}{\partial p_{2}}\\
\vdots  & \vdots  & \ddots  & \vdots \\
\frac{\partial x_{1}}{\partial p_{m}} & 
\frac{\partial x_{2}}{\partial p_{m}} & 
\cdots  & \frac{\partial x_{n}}{\partial p_{m}}
\end{array}\right]
\end{equation}
 

\subsection{Identity Transform}
\label{sec:IdentityTransform}
\begin{array}{rr}
\bf{Behavior} & 
Maps every point to itself, every vector to itself and every covariant vector to itself. \\
\bf{No. of parameters} & 
0 \\
\bf{Parameter Ordering} & 
\\
\bf{Restrictions} &
Only defined when the input and output space has the same number of dimensions. \\
\end{array}


\subsection{Translation Transform}
\label{sec:TranslationTransform}
\begin{array}{rr}
\bf{Behavior} & 
Represents a simple translation of points in the input space
and has no effect on vectors or covariant vectors. \\
\bf{No. of parameters} & 
Same as the input space dimension.\\
\bf{Parameter Ordering} & 
The i-th parameter represents the translation in the i-th dimension. \\
\bf{Restrictions} &
Only defined when the input and output space has the same number of dimensions. \\
\end{array}
 
\subsection{ScaleTransform}
\label{sec:ScaleTransform}
\begin{array}{rr}
\bf{Behavior} & 
Represents a simple scaling of the vector space. Each componet of a point, vector
or covariant vector is multiplied by the user defined scaling factor.\\
\bf{No. of parameters} & 
Same as the input space dimension. \\
\bf{Parameter Ordering} & 
The i-th parameter represents the scaling in the i-th dimension. \\
\bf{Restrictions} &
Only defined when the input and output space has the same number of dimensions. \\
\end{array}

\subsection{Euler2DTransform}
\label{sec:Euler2DTransform}
\begin{array}{rr}
\bf{Behavior} & 
Represents a rotation and a translation. Note that the translation
componet has no effect on the transformation of vectors and covariant vectors. \\
\bf{No. of parameters} & 
3\\
\bf{Parameter Ordering} & 
The first parameter is the angle in radian and the last two parameters
are the translation in each each dimension. \\
\bf{Restrictions} &
Only defined for two-dimensional input and output spaces. \\
\end{array}

\subsection{Similarity2DTransform}
\label{sec:Similarity2DTransform}
\begin{array}{rr}
\bf{Behavior} & 
Represents a rotation, homogenous scaling and a translation. Note that the translation
componet has no effect on the transformation of vectors and covariant vectors. \\
\bf{No. of parameters} & 
4\\
\bf{Parameter Ordering} & 
The first parameter is the angle in radian, the second the scaling factor for all
dimension and the last two parameters are the translation in each each dimension. \\
\bf{Restrictions} &
Only defined for two-dimensional input and output spaces.
\end{array}

\subsection{VersorRigid3DTransform}
%\subsection{VersorTransform}

\subsection{QuaternionRigidTransform}

\subsection{AffineTransform}



