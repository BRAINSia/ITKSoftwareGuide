%
%
%  This file in inserted in the Filtering.tex file.
%
%

The drawback of image denoising (smoothing) is that it tends to blur away the
sharp boundaries in the image that help to distinguish between the
larger-scale anatomical structures that one is trying to characterize (which
also limits the size of the smoothing kernels in most applications).  Even in
cases where smoothing does not obliterate boundaries, it tends to distort the
fine structure of the image and thereby changes subtle aspects of the
anatomical shapes in question.

Perona and Malik \cite{Perona1990} introduced an alternative to
linear-filtering that they called \emph{anisotropic diffusion}.  Anisotropic
diffusion is closely related to the earlier work of Grossberg
\cite{Grossberg1984}, who used similar nonlinear diffusion processes to model
human vision.  The motivation for anisotropic diffusion (also called
\emph{nonuniform} or \emph{variable conductance} diffusion) is that a Gaussian
smoothed image is a single time slice of the solution to the heat equation, 
that has the original image as its initial conditions.  Thus, the solution to
\begin{equation} \frac{\partial g(x, y, t) }{\partial t} = \nabla \cdot \nabla
g(x, y, t), \end{equation} where $g(x, y, 0) = f(x, y)$ is the input image, is
$g(x, y, t) = G(\sqrt{2t}) \otimes f(x, y)$, where $G(\sigma)$ is a Gaussian
with standard deviation $\sigma$.  

Anisotropic diffusion includes a variable conductance term that, in turn,
depends on the differential structure of the image.  Thus, the variable
conductance can be formulated to limit the smoothing at ``edges'' in images, as
measured by high gradient magnitude, for example. \begin{equation} g_{t} = \nabla \cdot
c(\left| \nabla g \right|) \nabla g, \label{eq:aniso} \end{equation} where, for
notational convenience, we leave off the independent parameters of $g$ and use
the subscripts with respect to those parameters to indicate partial
derivatives.  The function $c(|\nabla g|)$ is a fuzzy cutoff that reduces the
conductance at areas of large $|\nabla g|$, and can be any one of a number of
functions.  The literature has shown \begin{equation} c(|\nabla g|) =
e^{-\frac{|\nabla g|^{2}}{2k^{2}}} \end{equation} to be quite effective.
Notice that conductance term introduces a free parameter $k$, the {\em
conductance parameter}, that controls the sensitivity of the process to edge
contrast.  Thus, anisotropic diffusion entails two free parameters: the
conductance parameter, $k$, and the time parameter, $t$, that is analogous to
$\sigma$, the effective width of the filter when using Gaussian kernels.

Equation \ref{eq:aniso} is a nonlinear partial differential equation that can
be solved on a discrete grid using finite forward differences.  Thus, the
smoothed image is obtained only by an iterative process, not a convolution or
non-stationary, linear filter.  Typically, the number of iterations required
for practical results are small, and large 2D images can be processed in
several tens of seconds using carefully written code running on modern, general
purpose, single-processor computers.  The technique applies readily and
effectively to 3D images, but requires more processing time.

In the early 1990's several research groups \cite{Gerig1991,Whitaker1993d}
demonstrated the effectiveness of anisotropic diffusion on medical images.  In
a series of papers on the subject
\cite{Whitaker1993,Whitaker1993b,Whitaker1993c,Whitaker1993d,Whitaker-thesis,Whitaker1994},
Whitaker described a detailed analytical and empirical analysis, introduced a
smoothing term in the conductance that made the process more robust, invented a
numerical scheme that virtually eliminated directional artifacts in the
original algorithm, and generalized anisotropic diffusion to vector-valued
images, an image processing technique that can be used on vector-valued medical
data (such as the color cryosection data of the Visible Human Project).

For a vector-valued input $\vec{F}:U \mapsto \Re^{m}$ the process takes the
form \begin{equation} \vec{F}_{t} = \nabla \cdot c({\cal D}\vec{F}) \vec{F},
\label{eq:vector_diff} \end{equation} where ${\cal D}\vec{F}$ is a {\em
dissimilarity} measure of $\vec{F}$, a generalization of the gradient magnitude
to vector-valued images, that can incorporate linear and nonlinear coordinate
transformations on the range of $\vec{F}$.  In this way, the smoothing of the
multiple images associated with vector-valued data is coupled through the
conductance term, that fuses the information in the different images.  Thus
vector-valued, nonlinear diffusion can combine low-level image features (e.g.
edges) across all ``channels'' of a vector-valued image in order to preserve or
enhance those features in all of image ``channels''.

Vector-valued anisotropic diffusion is useful for denoising data from devices
that produce multiple values such as MRI or color photography.  When performing
nonlinear diffusion on a color image, the color channels are diffused
separately, but linked through the conductance term. Vector-valued diffusion it
is also useful for processing registered data from different devices or for
denoising higher-order geometric or statistical features from scalar-valued
images \cite{Whitaker1994,Yoo1993}.

The output of anisotropic diffusion is an image or set of images that
demonstrates reduced noise and texture but preserves, and can also enhance,
edges.  Such images are useful for a variety of  processes including
statistical classification, visualization, and geometric feature extraction.
Previous work has shown \cite{Whitaker-thesis} that anisotropic diffusion, over
a wide range of conductance parameters, offers quantifiable advantages over
linear filtering for edge detection in medical images.

Since the effectiveness of nonlinear diffusion was first demonstrated, numerous
variations of this approach have surfaced in the literature \cite{Romeny1994}.
These include alternatives for constructing dissimilarity measures
\cite{Sapiro1996}, directional (i.e., tensor-valued) conductance terms
\cite{Weickert1996,Alvarez1994} and level set interpretations
\cite{Whitaker2001}.
