
\chapter{DataRepresentation}
\label{sec:DataRepresentation}


This chapter introduces the basic classes responsible
for carrying data in ITK. The most common classes are the
\code{itk::Image},  the \code{itk::Mesh} and the \code{itk::PointSet}.

\section{Image}
\label{sec:ImageSection}

The Image class follows the spirit of Generic Programming where
types are separated from the algorithmic behavior of the class.
ITK supports images with any pixel type and any spatial dimension.

\subsection{Creating an Image}\label{sec:CreatingAnImageSection}


\input Image1.tex

In practice it is rare to allocate and initialize an image directly.
Typically the image is read from a source like a file or a data acquisition
card. The following example illustrates how an image can be read from
a file.




\subsection{Reading an Image from a file}
\label{sec:ReadingImageFromFile}

\input Image2.tex





\subsection{Accessing pixel data}
\label{sec:AccessingImagePixelData}

\input Image3.tex




\subsection{Defining Origin and Spacing}
\label{sec:DefiningImageOriginAndSpacing}

\input Image4.tex


\subsection{RGB Images}
\label{sec:DefiningRGBImages}

\input RGBImage.tex


\subsection{Vector Images}
\label{sec:DefiningVectorImages}

\input VectorImage.tex



\section{PointSet}
\label{PointSetSection}

\subsection{Creating a PointSet}
\label{sec:CreatingAPointSet}

\input PointSet1.tex



\subsection{Getting access to points}
\label{sec:GettingAccessToPointsInThePointSet}

\input PointSet2.tex



\subsection{Getting access to data in points}
\label{sec:GettingAccessToDataInThePointSet}

\input PointSet3.tex



\subsection{RGB as pixel type}
\label{sec:PointSetWithRGBAsPixelType}

\input RGBPointSet.tex




\subsection{Vectors as pixelType}
\label{sec:PointSetWithVectorsAsPixelType}

\input PointSetWithVectors.tex



\subsection{Normals as pixelType}
\label{sec:PointSetWithCovariantVectorsAsPixelType}

\input PointSetWithCovariantVectors.tex




\section{Mesh}\label{MeshSection}

\subsection{Creating a Mesh}
\label{sec:CreatingAMesh}

\input Mesh1.tex


\subsection{Inserting Cells}
\label{sec:InsertingCellsInMesh}

\input Mesh2.tex


\subsection{Managing Data in Cells}
\label{sec:ManagingCellDataInMesh}

\input Mesh3.tex


\subsection{Customizing the Mesh}
\label{sec:CustomizingTheMesh}

\input MeshTraits.tex


\subsection{Topoloy and the K-Complex}
\label{sec:MeshKComplex}

\input MeshKComplex.tex


\subsection{Iterating Throught Cells}
\label{sec:MeshCellsIteration}

\input MeshCellsIteration.tex


\subsection{Visiting Cells}
\label{sec:MeshCellVisitor}

\input MeshCellVisitor.tex


\subsection{More on Visiting Cells}
\label{sec:MeshCellVisitorMultipleType}

\input MeshCellVisitor2.tex


