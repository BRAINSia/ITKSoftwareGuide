
\index{itk::MultiResolutionPyramidImageFilter|textbf}

In ITK, the \code{MultiResolutionPyramidImageFilter} can be used to create a
sequence of down-sampled version of the input image.  The down-sampling is done
according to a user defined multi-resolution schedule. The schedule is
specified as an \code{Array2D<int>} containing shrink factors for each
multi-resolution level (rows) for each dimension (columns). For example,

\begin{verbatim}
8 4 4
4 4 2
\end{verbatim}

is a schedule for a three dimensional image for two multi-resolution levels. 
In the first (coarsest) level, the image is reduced by a factor of 8 
in the column dimension, factor of 4 in the row dimension and a factor
of 4 in the slice dimension. In the second level, the image reduced
by a factor of 4 in the column dimension, 4 in the row dimension and
2 in the slice dimension.

\index{itk::MultiResolutionPyramidImageFilter!SetNumberOfLevels()}

The method \code{SetNumberOfLevels()} is used to set the number of
resolution levels in the pyramid. This method will allocate memory
for the schedule and generates a default table with the starting
(coarsest) shrink factors for all dimension set to $(M-1)^2$, 
where $M$ is the number of levels. All factors are halved for
all subsequent levels. For example, if we set the number of levels
to 4, the default schedule is then:

\begin{verbatim}
8 8 8
4 4 4
2 2 2
1 1 1
\end{verbatim}

\index{itk::MultiResolutionPyramidImageFilter!GetSchedule()}
\index{itk::MultiResolutionPyramidImageFilter!SetSchedule()}
\index{itk::MultiResolutionPyramidImageFilter!SetStartingShrinkFactors()}

The user can get a copy of the schedule using method \code{GetSchedule()},
make modifications and reset it using method \code{SetSchedule()}.
Alternatively, a user can create a default table by specifying the
starting (coarsest) shrink factors using method 
\code{SetStartingShrinkFactors()}. The factors for the subsequent
levels are generated by halving the factor or setting to one, 
depending on which is larger. For example, for a 4 level pyramid
and starting factors of 8, 8 and 4, the generated schedule would be:

\begin{verbatim}
8 8 4
4 4 2
2 2 1
1 1 1
\end{verbatim}

When this filter is triggered by \code{Update()}, $M$ outputs are produced
where the $m$-th output correspond to the $m$-th level of the pyramid.
To generate these images, Gaussian smoothing is first performed using a
\code{DiscreteGaussianImageFilter} with the variance set to $(s/2)^2$,
where $s$ is the shrink factor. The smoothed images are then sub-sampled using
a \code{ShrinkImageFilter}.
