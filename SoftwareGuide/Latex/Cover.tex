\section*{About the cover}

This section describes the cover of this book. The task of designing the cover
was suppossed to be an easy one. However, finding an image that will honor the
capabilities of the toolkit end up being a challenging one. Given that the
origins of the Insight toolkit are rooted in the Visible Human Project it
seemed appropriate to make the image in the cover an application of the toolkit
to the segmentation and registration of the Visible Human data set. In fact the
more recently acquired Visible Woman datasets were used for this cover.  Both
the RGB cryosections and the CT scans were combined in the same scene.

\subsection*{Removing the Gel}

During the freezing process the body of the Visible Woman was inmersed in a
block of gel. This Gel appears as a blue substance in the cryogenic data.  In
order to remove the Get, the joint histogram of RGB values in the data set was
computed. This resulted in an 3D image of 256x256x256 pixels. The histogram
image was visualized in VolView. The cluster corresponding to the statistical
distribution of blue values was identified visually, and a separatrix plane was
manually defined in RGB space.

The equation of this plane was subsequently used for discriminating pixels in
the gel from pixels in the anatomical structures of the body. The Gel pixels
were set to null values and the RGB values on the body were preserved.

\subsection*{The Skin}

The skin was easy to segment once the Gel was removed. A simple region growing
algorithm was used providing seed points in the region previously occupied by
the Gel and now set to zero values. 

An Antialiasing filter was applied in order to generate an image of pixel type
float where the surface was represented by the zero set. This data set was
exported to VTK where a contouring filter was used in order to extract a
geometrical representation of the surface and introduce it in the VTK
visualization pipeline.


\subsection*{The Brain}

The visible part of the brain represents the surface of the gray matter.  The
brain was segmented using the Vector version of the confidence connected image
filter.  This filter implements a region growing algorithm that starting from a
set of seed points add neighbor pixels subject to a condition of homogeneity.

The set of sparse points obtained from the region growin algorithm was passed
through a mathematical morphology dilation in order to close holes and then
through a binary median filter. The binary median filter has the outstanding
characteristic of being very simple in implementation by applying a
sophisticated effect on the image. Qualitatively it is equivalent to a
curvature flow evolution of the iso-contouors. In fact the BinaryMedian filter
as implemented in ITK is equivalent to the Majority Filter which belong to the
family of Voting filter classified as a subset of the \emph{Larger than Life}
cellular automata, 

Finaly, the volume resulting from the median filter was passed through the 
antialiasin image filter.


\subsection*{The Neck Mucles}

The neck muscles are not perfectly segmented. Indeed, the current surface is a
fusion of muscles, blood vessels and other structures. The segmentation was
performed using the VectorConfidenceConnected filter on the cryogenic dataset.
A number of about 60 seed points were manually selected and then passed to this
filter as input.

The binary mask produced by the VectorConfidenceConnected was dilated with a 
mathematical morphology filter and smoothed with the BinaryMedian filter. The 
AntialiasImageFilter was used at the end to reduce the pixelization effects
previos to the extraction of the iso-surface with the vtkContourFilter.  



\subsection*{The Skull}

The Skull was segmented from the CT data set and registered to the cryogenic
data. The segmentation was performed by simple thresholding, which was good
enough for producing an appealing surface for the cover. This means that most
of the bone structures are actually fused together. This includes the jaw bone,
and the cervical vertebrae.

\subsection*{The Eye} 

The eye was charged with symbolism in this image. In part because the
motivation for the development of the toolkit was the analysis of the Visible
Human data, and in part because the toolkit was named \emph{Insight}.

The first step in processing the Eye was to extract a sub-image of
$60\times60\times60$ pixels centered around the eyeball from the RGB Cryogenic
data set.

This small volume was then processed with the Vector Gradient Anisotropic
Diffusion filter in order to increase the homogeneity of the pixels in the 
eyeball.

The smoothed volume was segmented using the VectorConfidenceConnected filter
and a number of 10 seed points. The resulting binary mask was dilated with 
a mathematical morphology filter with a structuring element of radius one, then
smoothed with a binary mean image filter which is equivalent to Majority voting
cellular automata. Finally the mask was processed with the AntialiasImageFilter
in order to generate a float image with the eyebal contour embebded as a zero
set.


\subsection*{Visualization}

The visualization of the segmentation was done by passing all the binary masks
through the itkAntialiasImageFilter, generating iso-contours with VTK filters,
and then setting up a VTK Tcl script.

The skin surface was clipped using the ClippPolyDataFilter and a vtkCylinder
as clipping function. The vtkWindowToImageFilter proved to be quite useful for
generating the final high resolution rendering of the scene (3000x3000 pixels).


\subsection*{Cosmetic Postprocessing}

We have to confess that we used Adobe Photoshop in the cosmetic post-processing of
the image. In particular, the background of the image was color-tunned using
photoshop's color selection. No hard feelings about that...

The overall composition of the cover was done with Photoshop and facilitated
the production of a high quality rendering.

