\chapter*{The Cover}

This section describes the cover of this book. The task of designing the cover
was suppossed to be an easy one. 

\section*{Removing the Gel}
During the freezing process the body of the Visible Woman was inmersed in a
block of gel. This Gel appears as a blue substance in the cryogenic data.

\section*{The Skin}
The skin was easy to segment once the Gel was removed. A simple region growing
algorithm was used providing seed points in the region previously occupied by
the gel. 

An Antialiasing filter was applied in order to generate an image of pixel type
float where the surface was represented by the zero set. This data set was
exported to VTK where a contouring filter was used in order to extract a
geometrical representation of the surface and introduce it in the VTK
visualization pipeline.


\section*{The Brain}
The visible part of the brain represents the surface of the gray matter.  The
brain was segmented using the Vector version of the confidence connected image
filter.  This filter implements a region growing algorithm that starting from a
set of seed points add neighbor pixels subject to a condition of homogeneity.

The set of sparse points obtained from the region growin algorithm was passed
through a mathematical morphology dilation in order to close holes and then
through a binary median filter. The binary median filter has the outstanding
characteristic of being very simple in implementation by applying a
sophisticated effect on the image. Qualitatively it is equivalent to a
curvature flow evolution of the iso-contouors. In fact the BinaryMedian filter
as implemented in ITK is equivalent to the Majority Filter which belong to the
family of Voting filter classified as a subset of the \emph{Larger than Life}
cellular automata, 

Finaly, the volume resulting from the median filter was passed through the 
antialiasin image filter.


\section*{The Neck Mucles}

\section*{The Skull}
The Skull was segmented from the CT data set and registered to the cryogenic data.

\section*{The Eye} 
The eye was charged with symbolism in this image. In part because the
motivation for the development of the toolkit was the analysis of the Visible
human data, and in part because the toolkit was names \emph{Insight}.

The first step in processing the Eye was to extract a sub-image of
$60\times60\times60$ pixels centered around the eyeball.

This small volume was then processed with the Vector Gradient Anisotropic
Diffusion filter in order to increase the homogeneity of the pixels in the 
eyeball.

The smoothed volume was segmented using the VectorConfidenceConnected filter
and a number of 10 seed points. The resulting binary mask was dilated with 
a mathematical morphology filter with a structuring element of radius one, then
smoothed with a binary mean image filter which is equivalent to Majority voting
cellular automata. Finally the mask was processed with the AntialiasImageFilter
in order to generate a float image with the eyebal contour embebded as a zero
set.


\section*{Visualization}

The visualization of the segementation was done by passing all the binary 
masks through the AntialiasImageFilter, generating iso-contours with VTK
filters, and then setting up a VTK script.

The skin surface was clipped using the ClippPolyDataFilter and a vtkCylinder
as function.
