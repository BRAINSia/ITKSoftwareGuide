
\chapter{DataRepresentation}
\label{sec:DataRepresentation}


This chapter introduces the basic classes responsible
for carrying data in ITK. The most common classes are the
itk::Image,  the itk::Mesh and the itk::PointSet.

\section{Image}
\label{sec:ImageSection}

The Image class follows the spirit of Generic Programming where
types are separated from the algorithmic behavior of the class.
ITK supports images with any pixel type and any spatial dimension.

\subsection{Creating an Image}\label{sec:CreatingAnImageSection}

%
% The following file is automatically generated
% by a perl script from the original cxx sources
% in the Insight/Examples directory
%
\input Image1.tex

In practice it is rare to allocate and initialize an image directly.
Typically the image is read from a source like a file or a data acquisition
card. The following example illustrates how an image can be read from
a file.




\subsection{Reading an Image from a file}
\label{sec:ReadingImageFromFile}
%
% The following file is automatically generated
% by a perl script from the original cxx sources
% in the Insight/Examples directory
%
\input Image2.tex





\subsection{Accessing pixel data}
\label{sec:AccessingImagePixelData}
%
% The following file is automatically generated
% by a perl script from the original cxx sources
% in the Insight/Examples directory
%
\input Image3.tex




\subsection{Defining Origin and Spacing}
\label{sec:DefiningImageOriginAndSpacing}
%
% The following file is automatically generated
% by a perl script from the original cxx sources
% in the Insight/Examples directory
%
\input Image4.tex


\subsection{RGB Images}
\label{sec:DefiningRGBImages}
%
% The following file is automatically generated
% by a perl script from the original cxx sources
% in the Insight/Examples directory
%
\input RGBImage.tex


\subsection{Vector Images}
\label{sec:DefiningVectorImages}
%
% The following file is automatically generated
% by a perl script from the original cxx sources
% in the Insight/Examples directory
%
\input VectorImage.tex



\section{PointSet}\label{PointSetSection}

\subsection{Creating a PointSet}\label{sec:CreatingAPointSet}

%
% The following file is automatically generated
% by a perl script from the original cxx sources
% in the Insight/Examples directory
%
\input PointSet1.tex





\section{Mesh}\label{MeshSection}


